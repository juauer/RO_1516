% !TEX TS-program = pdflatex
% !TEX encoding = UTF-8 Unicode

\documentclass[a4paper, titlepage=false, parskip=full-, 10pt]{scrartcl}

\usepackage[utf8]{inputenc}
\usepackage[T1]{fontenc}
\usepackage[english, ngerman]{babel}
\usepackage{babelbib}
\usepackage{hyperref}
\usepackage{listings}
\usepackage{framed}
\usepackage{color}
\usepackage{graphicx}
\usepackage[normalem]{ulem}
\usepackage{cancel}
\usepackage{array}
\usepackage{amsmath}
\usepackage{amssymb}
\usepackage{amsthm}
\usepackage{algorithm}
\usepackage{algorithmic}
\usepackage{geometry}
\usepackage{subfigure}
\geometry{a4paper, top=20mm, left=35mm, right=25mm, bottom=40mm}

\newcounter{tasknbr}
\setcounter{tasknbr}{1}
\newenvironment{task}[1]{{\bf Aufgabe \arabic {tasknbr}\stepcounter{tasknbr}} (#1):\begin{enumerate}}{\end{enumerate}}
\newcommand{\subtask}[1]{\item[#1)]}

% Listings -----------------------------------------------------------------------------
\definecolor{red}{rgb}{.8,.1,.2}
\definecolor{blue}{rgb}{.2,.3,.7}
\definecolor{lightyellow}{rgb}{1.,1.,.97}
\definecolor{gray}{rgb}{.7,.7,.7}
\definecolor{darkgreen}{rgb}{0,.5,.1}
\definecolor{darkyellow}{rgb}{1.,.7,.3}
\lstloadlanguages{C++,[Objective]C,Java}
\lstset{
escapeinside={§§}{§§},
basicstyle=\ttfamily\footnotesize\mdseries,
columns=fullflexible,
keywordstyle=\bfseries\color{blue},
commentstyle=\color{darkgreen},      
stringstyle=\color{red},
numbers=left,
numberstyle=\ttfamily\scriptsize\color{gray},
breaklines=true,
showstringspaces=false,
tabsize=4,
captionpos=b,
float=htb,
frame=tb,
frameshape={RYR}{y}{y}{RYR},
rulecolor=\color{black},
xleftmargin=15pt,
xrightmargin=4pt,
aboveskip=\bigskipamount,
belowskip=\bigskipamount,
backgroundcolor=\color{lightyellow},
extendedchars=true,
belowcaptionskip=15pt}

%% Enter current values here: %%
\newcommand{\lecture}{Robotik WS15/16}
\newcommand{\tutor}{}
\newcommand{\assignmentnbr}{7}
\newcommand{\students}{Julius Auer, Thomas Tegethoff}
%%-------------------------------------%%

\begin{document}  
{\small \textsl{\lecture \hfill \tutor}}
\hrule
\begin{center}
\textbf{Übungsblatt \assignmentnbr}\\
[\bigskipamount]
{\small \students}
\end{center}
\hrule

\begin{task}{Control}
\subtask{a}
Im Moment ist das so:

- Ich berechne den Fehler - wie vorgeschlagen - aus dem Winkel\\
- Ich stecke ihn in den Regler\\
- Ich plotte 1) y, 2) den Fehler und 3) den Regelausgang

Im Header stehen alle relevanten Konstanten, außer p, i, m, d, x0, y0 und können dort angepasst werden.

x0, y0 (Startposition) müssen in launch/gazebo.launch angepasst werden

Zum starten:\\
roslaunch u07 gazebo.launch\\
roslaunch u07 pid.launch p:=<p> i:=<i> m:=<m> d:=<d>

Wobei:\\
p: der P-Anteil (double)\\
i: der I-Anteil (double)\\
d: der D-Anteil (double)\\
m: die Länge der Liste für den I-Anteil (Gedächtnis) (int)
 
\subtask{b}
Es müssen nun gute Werte gefunden und geplottet werden. Aktuelles Problem: es oszilliert nicht:

Fährt man langsam, ist der primitivste P-Regler schon perfekt (keine Oszillation), fährt man hingegen schnell, geht das Auto kaputt (Überschlag).

Ich vermute, man {\bf muss} msg.drive.steering\_angle\_velocity setzen (ich nehme derzeit immer Maximum). Ich mache aber für heute erst mal Schluss.

Viel Spass beim rumspielen :)
\end{task}
\end{document}