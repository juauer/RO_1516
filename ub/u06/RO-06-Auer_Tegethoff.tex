% !TEX TS-program = pdflatex
% !TEX encoding = UTF-8 Unicode

\documentclass[a4paper, titlepage=false, parskip=full-, 10pt]{scrartcl}

\usepackage[utf8]{inputenc}
\usepackage[T1]{fontenc}
\usepackage[english, ngerman]{babel}
\usepackage{babelbib}
\usepackage{hyperref}
\usepackage{listings}
\usepackage{framed}
\usepackage{color}
\usepackage{graphicx}
\usepackage[normalem]{ulem}
\usepackage{cancel}
\usepackage{array}
\usepackage{amsmath}
\usepackage{amssymb}
\usepackage{amsthm}
\usepackage{algorithm}
\usepackage{algorithmic}
\usepackage{geometry}
\usepackage{subfigure}
\geometry{a4paper, top=20mm, left=35mm, right=25mm, bottom=40mm}

\newcounter{tasknbr}
\setcounter{tasknbr}{1}
\newenvironment{task}[1]{{\bf Aufgabe \arabic {tasknbr}\stepcounter{tasknbr}} (#1):\begin{enumerate}}{\end{enumerate}}
\newcommand{\subtask}[1]{\item[#1)]}

% Listings -----------------------------------------------------------------------------
\definecolor{red}{rgb}{.8,.1,.2}
\definecolor{blue}{rgb}{.2,.3,.7}
\definecolor{lightyellow}{rgb}{1.,1.,.97}
\definecolor{gray}{rgb}{.7,.7,.7}
\definecolor{darkgreen}{rgb}{0,.5,.1}
\definecolor{darkyellow}{rgb}{1.,.7,.3}
\lstloadlanguages{C++,[Objective]C,Java}
\lstset{
escapeinside={§§}{§§},
basicstyle=\ttfamily\footnotesize\mdseries,
columns=fullflexible,
keywordstyle=\bfseries\color{blue},
commentstyle=\color{darkgreen},      
stringstyle=\color{red},
numbers=left,
numberstyle=\ttfamily\scriptsize\color{gray},
breaklines=true,
showstringspaces=false,
tabsize=4,
captionpos=b,
float=htb,
frame=tb,
frameshape={RYR}{y}{y}{RYR},
rulecolor=\color{black},
xleftmargin=15pt,
xrightmargin=4pt,
aboveskip=\bigskipamount,
belowskip=\bigskipamount,
backgroundcolor=\color{lightyellow},
extendedchars=true,
belowcaptionskip=15pt}

%% Enter current values here: %%
\newcommand{\lecture}{Robotik WS15/16}
\newcommand{\tutor}{}
\newcommand{\assignmentnbr}{6}
\newcommand{\students}{Julius Auer, Thomas Tegethoff}
%%-------------------------------------%%

\begin{document}  
{\small \textsl{\lecture \hfill \tutor}}
\hrule
\begin{center}
\textbf{Übungsblatt \assignmentnbr}\\
[\bigskipamount]
{\small \students}
\end{center}
\hrule

\begin{task}{Inverse Kinematik}
\item[]
In der Vorlesung wurde die Jocobi-Matrix $J$ bereits hergeleitet:
\begin{align*}
J&=\begin{pmatrix}-y&-l_2\cdot s_{12}\\x&l_2\cdot c_{12}\end{pmatrix}
\end{align*}

Da die Jacobi-Matrix hier immer hübsch quadratisch ist, kann die Inverse direkt über die geschlossene Form berechnet werden:

\begin{align*}
J^{-1}&=\frac{1}{x\cdot l_2\cdot s_{12}-y\cdot l_2\cdot c_{12}}\cdot\begin{pmatrix}l_2\cdot c_{12}&l_2\cdot s_{12}\\-x&-y\end{pmatrix}
\end{align*}

Der Roboter hat allerdings anders angeordnete Achsen, so dass die Matrix noch einmal angepasst werden muss, zu:

\begin{align*}
J^{-1}&=\frac{1}{z\cdot l_2\cdot s_{12}+y\cdot l_2\cdot c_{12}}\cdot\begin{pmatrix}-l_2\cdot s_{12}&l_2\cdot c_{12}\\-y&-z\end{pmatrix}
\end{align*}

Es seien nun: $l_1,l_2$ fest, $tf(\Theta_1,\Theta_2):\mathbb{R}\times\mathbb{R}\rightarrow\mathbb{R}^2$ eine Funktion um die aktuelle Pose $\overrightarrow{x}$ zu berechnen, $j(\Theta_1,\Theta_2):\mathbb{R}^2\rightarrow\mathbb{R}^{2\times 2}$ eine Funktion um wie gezeigt $J^{-1}$ zu erzeugen und $I\in\mathbb{N}$ die Anzahl der Iterationen. Dann lässt sich die inverse Kinematik folgenderweise berechnen:

\begin{enumerate}
\item[(1)]Initiale Winkel: $\Theta_1:=-\frac{1}{8}\cdot\pi,\Theta_2:=\frac{3}{4}\cdot\pi$
\item[(2)]End-Effektor-Position: $\overrightarrow{x}=tf(\Theta_1,\Theta_2)$
\item[(3)]Ziel-Position: $\overrightarrow{x_t}$
\item[(4)]Für alle $i\in I$:
\begin{enumerate}
\item[(5)]Prüfe und behandle ggf. Singularität (tritt hier mit hoher Wahrscheinlichkeit nicht auf, falls Anfangsposition günstig gewählt)
\item[(6)]$\begin{pmatrix}\Theta_1\\\Theta_2\end{pmatrix}+=\frac{i}{|I]}\cdot j(\Theta_1,\Theta_2)\cdot (\overrightarrow{x_t}-\overrightarrow{x})$
\item[(7)]$\overrightarrow{x}=tf(\Theta_1,\Theta_2)$
\end{enumerate}
\end{enumerate}

Genauso implementiert liefert das die folgenden Ergebnisse:

\begin{figure}[!htpb]
\begin{center}
\subfigure[Plot]{
  \includegraphics[width=0.9\linewidth]{capture_1-1}
}
\subfigure[Trajektorie]{
  \includegraphics[width=0.9\linewidth]{capture_1-2}
}
\end{center}
\caption{1 Iteration}
\end{figure}

\subtask{b}
\begin{figure}[!htpb]
\begin{center}
\subfigure[Plot]{
  \includegraphics[width=0.9\linewidth]{capture_1-3}
}
\subfigure[Trajektorie]{
  \includegraphics[width=0.9\linewidth]{capture_1-4}
}
\end{center}
\caption{10 Iteration}
\end{figure}

\subtask{c}
\begin{figure}[!htpb]
\begin{center}
\subfigure[Plot]{
  \includegraphics[width=0.9\linewidth]{capture_1-5}
}
\subfigure[Trajektorie]{
  \includegraphics[width=0.9\linewidth]{capture_1-6}
}
\end{center}
\caption{100 Iteration}
\end{figure}

\subtask{d}
\begin{figure}[!htpb]
\begin{center}
\subfigure[Plot]{
  \includegraphics[width=0.9\linewidth]{capture_1-7}
}
\subfigure[Trajektorie]{
  \includegraphics[width=0.9\linewidth]{capture_1-8}
}
\end{center}
\caption{1000 Iteration}
\end{figure}
\end{task}

\newpage\newpage\newpage\newpage
\begin{task}{Null-Space}
\item[]
$
	nullspace \left(\begin{pmatrix} 
				0.5\\
				y
			\end{pmatrix}\right) = \{
				\begin{pmatrix}
					\Theta_{1}\\
					\Theta_{2}
				\end{pmatrix} | -q_{1} \le \Theta_{1} \le q_{1} \land \Theta_{2} = \begin{cases}
					q_{2}, \text{ falls } \Theta_{1} = \pm q_{1}\\
					\text{sonst}\\
				\end{cases}
			\}
$
\end{task}
\end{document}
