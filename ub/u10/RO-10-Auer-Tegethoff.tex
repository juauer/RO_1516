% !TEX TS-program = pdflatex
% !TEX encoding = UTF-8 Unicode

\documentclass[a4paper, titlepage=false, parskip=full-, 10pt]{scrartcl}

\usepackage[utf8]{inputenc}
\usepackage[T1]{fontenc}
\usepackage[english, ngerman]{babel}
\usepackage{babelbib}
\usepackage{hyperref}
\usepackage{listings}
\usepackage{framed}
\usepackage{color}
\usepackage{graphicx}
\usepackage[normalem]{ulem}
\usepackage{cancel}
\usepackage{array}
\usepackage{amsmath}
\usepackage{amssymb}
\usepackage{amsthm}
\usepackage{algorithm}
\usepackage{algorithmic}
\usepackage{geometry}
\usepackage{subfigure}
\geometry{a4paper, top=20mm, left=35mm, right=25mm, bottom=40mm}

\newcounter{tasknbr}
\setcounter{tasknbr}{1}
\newenvironment{task}[1]{{\bf Aufgabe \arabic {tasknbr}\stepcounter{tasknbr}} (#1):\begin{enumerate}}{\end{enumerate}}
\newcommand{\subtask}[1]{\item[#1)]}

% Listings -----------------------------------------------------------------------------
\definecolor{red}{rgb}{.8,.1,.2}
\definecolor{blue}{rgb}{.2,.3,.7}
\definecolor{lightyellow}{rgb}{1.,1.,.97}
\definecolor{gray}{rgb}{.7,.7,.7}
\definecolor{darkgreen}{rgb}{0,.5,.1}
\definecolor{darkyellow}{rgb}{1.,.7,.3}
\lstloadlanguages{C++,[Objective]C,Java}
\lstset{
escapeinside={§§}{§§},
basicstyle=\ttfamily\footnotesize\mdseries,
columns=fullflexible,
keywordstyle=\bfseries\color{blue},
commentstyle=\color{darkgreen},      
stringstyle=\color{red},
numbers=left,
numberstyle=\ttfamily\scriptsize\color{gray},
breaklines=true,
showstringspaces=false,
tabsize=4,
captionpos=b,
float=htb,
frame=tb,
frameshape={RYR}{y}{y}{RYR},
rulecolor=\color{black},
xleftmargin=15pt,
xrightmargin=4pt,
aboveskip=\bigskipamount,
belowskip=\bigskipamount,
backgroundcolor=\color{lightyellow},
extendedchars=true,
belowcaptionskip=15pt}

%% Enter current values here: %%
\newcommand{\lecture}{Robotik WS15/16}
\newcommand{\tutor}{}
\newcommand{\assignmentnbr}{10}
\newcommand{\students}{Julius Auer, Thomas Tegethoff}
%%-------------------------------------%%

\begin{document}  
{\small \textsl{\lecture \hfill \tutor}}
\hrule
\begin{center}
\textbf{Übungsblatt \assignmentnbr}\\
[\bigskipamount]
{\small \students}
\end{center}
\hrule

\begin{task}{Voronoi-Diagramme}
\item[]
Letztes Semester habe ich bereits für eine andere Lehrveranstaltung (Algorithmische Geometrie) Fortune's Sweep implementiert. Genau genommen haben wir dort den Algo nur theoretisch behandelt - Prof. Alt hielt eine Implementierung im Rahmen des Übungsbetriebs für zu aufwändig ;)\\
Ich hatte aber trotzdem Lust dazu, was sich nun auszahlt :)

Der Code ist diesmal in Java. Die Implementierung benötigt eine Anzahl Hilfsklassen (Kreise, Strahlen, Strecken, etc.) weshalb ich die gesamte Codebasis aus Algorithmische Geometrie mit einreichen muss. Solltet Ihr den Code lesen wollen müsst Ihr deshalb etwas suchen, am besten in \emph{geometry/algorithms/FortunesSweep}. Der Code ist mit im Jar.

Das abgegebene Jar ist ausführbar (erfordert Java 7) und visualisiert Fortune's Sweep für 80 zufällig generierte Punkte. Für Keyboard-Controls siehe stdout.

Für zehn zufällige Punkte zeigt Abbildung \ref{fig:1-1} die ersten vier Schritte des Algos und Abbildung \ref{fig:1-2} die letzten vier.

Die Plots sollten einigermaßen selbst-erklärend sein - ungewöhnlich sind vielleicht nur die Cyan-farbenen Linien, die für jedes Parabelsegment den Punkt visualisieren, der die zugehörige Parabel erzeugt hat. Man sieht so auch, wie viele Segmente einer Parabel noch ''im Rennen'' sind.

Es ist anzumerken, dass die Parabelsegmente hier nicht in einem Baum (sondern nur in einer Liste) gespeichert werden. Dadurch wird die worst-case Laufzeit quadratisch! Da hier die Visualisierung im Vordergrund stand und die Implementierung des Baums doch recht aufwändig gewesen wäre, haben wir das in Kauf genommen.

\begin{figure}[!htpb]
\centering
\subfigure[Schritt 1]{
  \includegraphics[width=0.6\linewidth]{capture_1-1}
}
\subfigure[Schritt 2]{
  \includegraphics[width=0.6\linewidth]{capture_1-2}
}
\subfigure[Schritt 3]{
  \includegraphics[width=0.6\linewidth]{capture_1-3}
}
\subfigure[Schritt 4]{
  \includegraphics[width=0.6\linewidth]{capture_1-4}
}
\caption{Die ersten Schritte von Fortune's Sweep}
\label{fig:1-1}
\end{figure}

\begin{figure}[!htpb]
\centering
\subfigure[Schritt n-3]{
  \includegraphics[width=0.6\linewidth]{capture_1-5}
}
\subfigure[Schritt n-2]{
  \includegraphics[width=0.6\linewidth]{capture_1-6}
}
\subfigure[Schritt n-1]{
  \includegraphics[width=0.6\linewidth]{capture_1-7}
}
\subfigure[Schritt n]{
  \includegraphics[width=0.6\linewidth]{capture_1-8}
}
\caption{Die letzten Schritte von Fortune's Sweep}
\label{fig:1-2}
\end{figure}
\end{task}

\newpage
\begin{task}{Potentialfelder}
\item[]

\end{task}
\end{document}