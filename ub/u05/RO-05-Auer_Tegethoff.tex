% !TEX TS-program = pdflatex
% !TEX encoding = UTF-8 Unicode

\documentclass[a4paper, titlepage=false, parskip=full-, 10pt]{scrartcl}

\usepackage[utf8]{inputenc}
\usepackage[T1]{fontenc}
\usepackage[english, ngerman]{babel}
\usepackage{babelbib}
\usepackage{hyperref}
\usepackage{listings}
\usepackage{framed}
\usepackage{color}
\usepackage{graphicx}
\usepackage[normalem]{ulem}
\usepackage{cancel}
\usepackage{array}
\usepackage{amsmath}
\usepackage{amssymb}
\usepackage{amsthm}
\usepackage{algorithm}
\usepackage{algorithmic}
\usepackage{geometry}
\usepackage{subfigure}
\geometry{a4paper, top=20mm, left=35mm, right=25mm, bottom=40mm}

\newcounter{tasknbr}
\setcounter{tasknbr}{1}
\newenvironment{task}[1]{{\bf Aufgabe \arabic {tasknbr}\stepcounter{tasknbr}} (#1):\begin{enumerate}}{\end{enumerate}}
\newcommand{\subtask}[1]{\item[#1)]}

% Listings -----------------------------------------------------------------------------
\definecolor{red}{rgb}{.8,.1,.2}
\definecolor{blue}{rgb}{.2,.3,.7}
\definecolor{lightyellow}{rgb}{1.,1.,.97}
\definecolor{gray}{rgb}{.7,.7,.7}
\definecolor{darkgreen}{rgb}{0,.5,.1}
\definecolor{darkyellow}{rgb}{1.,.7,.3}
\lstloadlanguages{C++,[Objective]C,Java}
\lstset{
escapeinside={§§}{§§},
basicstyle=\ttfamily\footnotesize\mdseries,
columns=fullflexible,
keywordstyle=\bfseries\color{blue},
commentstyle=\color{darkgreen},      
stringstyle=\color{red},
numbers=left,
numberstyle=\ttfamily\scriptsize\color{gray},
breaklines=true,
showstringspaces=false,
tabsize=4,
captionpos=b,
float=htb,
frame=tb,
frameshape={RYR}{y}{y}{RYR},
rulecolor=\color{black},
xleftmargin=15pt,
xrightmargin=4pt,
aboveskip=\bigskipamount,
belowskip=\bigskipamount,
backgroundcolor=\color{lightyellow},
extendedchars=true,
belowcaptionskip=15pt}

%% Enter current values here: %%
\newcommand{\lecture}{Robotik WS15/16}
\newcommand{\tutor}{}
\newcommand{\assignmentnbr}{5}
\newcommand{\students}{Julius Auer, Thomas Tegethoff}
%%-------------------------------------%%

\begin{document}  
{\small \textsl{\lecture \hfill \tutor}}
\hrule
\begin{center}
\textbf{Übungsblatt \assignmentnbr}\\
[\bigskipamount]
{\small \students}
\end{center}
\hrule

\begin{task}{Umzug nach ROS-Indigo}
\item[]
... ist schon letzte Woche erfolgt.
\end{task}

\begin{task}{Roboter-Simulator Gazebo}
\item[]

\begin{figure}[!htpb]
\centering
\includegraphics[width=0.8\linewidth]{capture_2-1}
\caption{Kamerabild (gazebo)}
\label{fig:2-1}
\end{figure}

\begin{figure}[!htpb]
\centering
\includegraphics[width=0.8\linewidth]{capture_2-2}
\caption{Punktwolke (rviz)}
\label{fig:2-2}
\end{figure}

\begin{figure}[!htpb]
\centering
\includegraphics[width=0.8\linewidth]{capture_2-3}
\caption{Fahrzeug (gazebo)}
\label{fig:2-3}
\end{figure}
\end{task}

\begin{task}{DH-Parameter}
\item[]
\begin{table}[!htpb]
\centering
\begin{tabular}{>{$}c<{$}|>{$}c<{$}>{$}c<{$}>{$}c<{$}>{$}c<{$}}
i	&	d	&	\theta	&	a	&	\alpha\\\hline
1	&	L_1	&	\theta_1	&	0	&	0\\
2	&	0	&	\theta_2	&	0	&	\frac{\pi}{2}\\
3	&	0	&	\theta_3	&	L_2	&	0
\end{tabular}
\end{table}

Config. shown: $\theta_1=160,\theta_2=45,\theta_3=20$

Transformation von \{3\} nach \{2\}:

\begin{align*}
T_{\{3\}\rightarrow\{2\}}=\begin{pmatrix}c\theta_3&-s\theta_3&0&L_2\cdot c\theta_3\\
s\theta_3&c\theta_3&0&L_2\cdot s\theta_3\\
0&0&1&0\\
0&0&0&1\end{pmatrix}
\end{align*}

Plausibilitäts-Test:\\
Seien: $\theta =\frac{\pi}{2},L_2=1,x_{\{3\}}=\begin{pmatrix}1\\1\\0\end{pmatrix}$\\
Erwartung: $T_{\{3\}\rightarrow\{2\}}\cdot x_{\{3\}}=\begin{pmatrix}-1\\2\\0\end{pmatrix}$

\begin{align*}
\begin{pmatrix}c\left(\frac{\pi}{2}\right)&-s\left(\frac{\pi}{2}\right)&0&c\left(\frac{\pi}{2}\right)\\
s\left(\frac{\pi}{2}\right)&c\left(\frac{\pi}{2}\right)&0&s\left(\frac{\pi}{2}\right)\\
0&0&1&0\\
0&0&0&1\end{pmatrix}\cdot\begin{pmatrix}1\\1\\0\\1\end{pmatrix}&=\begin{pmatrix}0-1+0+0\\1+0+0+1\\0+0+0+0\\0+0+0+1\end{pmatrix}=\begin{pmatrix}-1\\2\\0\\1\end{pmatrix}
\end{align*}

$\rightarrow$ Super.
\end{task}

\begin{task}{Jacobi-Matrix}
\subtask{a}

\subtask{b}

\subtask{c}

\end{task}
\end{document}