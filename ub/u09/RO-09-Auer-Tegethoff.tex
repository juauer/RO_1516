% !TEX TS-program = pdflatex
% !TEX encoding = UTF-8 Unicode

\documentclass[a4paper, titlepage=false, parskip=full-, 10pt]{scrartcl}

\usepackage[utf8]{inputenc}
\usepackage[T1]{fontenc}
\usepackage[english, ngerman]{babel}
\usepackage{babelbib}
\usepackage{hyperref}
\usepackage{listings}
\usepackage{framed}
\usepackage{color}
\usepackage{graphicx}
\usepackage[normalem]{ulem}
\usepackage{cancel}
\usepackage{array}
\usepackage{amsmath}
\usepackage{amssymb}
\usepackage{amsthm}
\usepackage{algorithm}
\usepackage{algorithmic}
\usepackage{geometry}
\usepackage{subfigure}
\geometry{a4paper, top=20mm, left=35mm, right=25mm, bottom=40mm}

\newcounter{tasknbr}
\setcounter{tasknbr}{1}
\newenvironment{task}[1]{{\bf Aufgabe \arabic {tasknbr}\stepcounter{tasknbr}} (#1):\begin{enumerate}}{\end{enumerate}}
\newcommand{\subtask}[1]{\item[#1)]}

% Listings -----------------------------------------------------------------------------
\definecolor{red}{rgb}{.8,.1,.2}
\definecolor{blue}{rgb}{.2,.3,.7}
\definecolor{lightyellow}{rgb}{1.,1.,.97}
\definecolor{gray}{rgb}{.7,.7,.7}
\definecolor{darkgreen}{rgb}{0,.5,.1}
\definecolor{darkyellow}{rgb}{1.,.7,.3}
\lstloadlanguages{C++,[Objective]C,Java}
\lstset{
escapeinside={§§}{§§},
basicstyle=\ttfamily\footnotesize\mdseries,
columns=fullflexible,
keywordstyle=\bfseries\color{blue},
commentstyle=\color{darkgreen},      
stringstyle=\color{red},
numbers=left,
numberstyle=\ttfamily\scriptsize\color{gray},
breaklines=true,
showstringspaces=false,
tabsize=4,
captionpos=b,
float=htb,
frame=tb,
frameshape={RYR}{y}{y}{RYR},
rulecolor=\color{black},
xleftmargin=15pt,
xrightmargin=4pt,
aboveskip=\bigskipamount,
belowskip=\bigskipamount,
backgroundcolor=\color{lightyellow},
extendedchars=true,
belowcaptionskip=15pt}

%% Enter current values here: %%
\newcommand{\lecture}{Robotik WS15/16}
\newcommand{\tutor}{}
\newcommand{\assignmentnbr}{9}
\newcommand{\students}{Julius Auer, Thomas Tegethoff}
%%-------------------------------------%%

\begin{document}  
{\small \textsl{\lecture \hfill \tutor}}
\hrule
\begin{center}
\textbf{Übungsblatt \assignmentnbr}\\
[\bigskipamount]
{\small \students}
\end{center}
\hrule

\begin{task}{A*-Suche}
\item[]
Da die Geschwindigkeit des Autos sowie die Schrittweite fest aber beliebig sind, haben wir einen kontinuierlichen Zustandsraum und lassen die CLOSED-Liste weg. Für den hier behandelten einfachen Fall wäre eine Implementierung für kontinuierliche Zustandsräume ein unverhältnismäßig großer Aufwand.

Die Auswahl der Nachbarn eines Knotens ist straight-forward: Knoten repräsentieren eine 3D-Pose des Autos, so dass es stets 3 Nachbarn gibt - je nachdem, ob man links / rechts / gar nicht einlenkt. Die neue Pose für den neuen Knoten ergibt sich daraus direkt.

Interessant ist die Wahl einer guten Heuristik. Jeder Satz der Aufgabenstellung scheint hier laut {\bf Dubin} zu rufen - da hier etwas Einfacheres auch nicht ausreichend wäre (vor Allem die Bewegung in unmittelbarer Nähe eines Hindernisses erfordert hier ein Modell, das die nicht-Holonomie des Agenten abbilden kann), der Bedarf von etwas Ambitionierterem (Reeds-Shepp) aber durch die Randbedingungen der Aufgabe ausgeschlossen werden kann, implementieren wir als Heuristik eben genau besagte Dubin-Kurven.

Als einfache, wirkungsvolle Optimierung verbessern wir die Situation beim Rangieren vor einem Hindernis durch Hinzufügen eines Potentialfeld-ähnlichen Abstoßungseffektes: besonders ungünstig für die Wegfindung ist das frontale ''auf ein Hindernis zuhalten'', bei dem sich das Auto anschließend in langen Dubin-Kurven ''verheddern'' kann. Dies wiegt umso schwerer, wenn aufgrund der fehlenden CLOSED-Liste Kreise auftreten können. Als einfache Lösung wird einer Pose bei der ein Hindernis ''vor'' dem Auto ist ein zusätzliches Gewicht gegeben, das mit dem Abstand ($L_2$) zum Hindernis skaliert. Ist das Hindernis nicht direkt ''vor'' dem Auto darf dieser Effekt natürlich nicht auftreten, so dass man nach wie vor ''nah'' am Hindernis vorbeifahren kann.

UPDATE: Eine zuerst testweise implementierte euklidsche Heuristik liefert schon akzeptable Ergebnisse (Stand: 19.12.). Ob wir jetzt noch einen Anreiz haben, das mit dem geschilderten, ambitionierteren Ansatz zu vergleichen? Schätze, ich mach' erstmal Pause ...

Es seien im Folgenden: $e_p=2.5$ der zulässige Positionfehler, $e_o=1$ der zulässige Orientierungsfehler, $d$ die Schrittweite (als Vielfaches von $\pi$), $r=4$ der Radius des Wendekreises, $(x_o,y_o)\in\{\mathbb{R}^2|15\le x_o\le20\wedge -5\le y_o\le 20\}$ das Hindernis, $((x_t,y_t),(ox_t,oy_t))$ die Zielpose und $|O_d|$ die Größe der OPEN-Liste bei Schrittweite $d$.

\newpage
\subtask{a}
\begin{align*}
((x_t,y_t),(ox_t,oy_t))&=((6,3),(0,1))\\
|O_1|&=3\\
|O_{0.5}|&=7
\end{align*}

\begin{figure}[!htpb]
\centering
\subfigure[$d=\pi$]{
  \includegraphics[width=0.3\linewidth]{capture_1-1}
}
\subfigure[$d=\frac{\pi}{2}$]{
  \includegraphics[width=0.3\linewidth]{capture_1-2}
}
\caption{Plot a}
\end{figure}

\subtask{b}
\begin{align*}
((x_t,y_t),(ox_t,oy_t))&=((0,5),(0,1))\\
|O_1|&=113\\
|O_{0.5}|&=3059
\end{align*}

\begin{figure}[!htpb]
\centering
\subfigure[$d=\pi$]{
  \includegraphics[width=0.3\linewidth]{capture_1-3}
}
\subfigure[$d=\frac{\pi}{2}$]{
  \includegraphics[width=0.3\linewidth]{capture_1-4}
}
\caption{Plot b}
\end{figure}

\newpage
\subtask{c}
\begin{align*}
((x_t,y_t),(ox_t,oy_t))&=((0,5),(1,0))\\
|O_1|&=13\\
|O_{0.5}|&=103
\end{align*}

\begin{figure}[!htpb]
\centering
\subfigure[$d=\pi$]{
  \includegraphics[width=0.3\linewidth]{capture_1-5}
}
\subfigure[$d=\frac{\pi}{2}$]{
  \includegraphics[width=0.3\linewidth]{capture_1-6}
}
\caption{Plot c}
\end{figure}

\subtask{d}
\begin{align*}
((x_t,y_t),(ox_t,oy_t))&=((30,15),(1,0))\\
|O_1|&=1300\\
|O_{0.5}|&=3073656
\end{align*}

\begin{figure}[!htpb]
\centering
\subfigure[$d=\pi$]{
  \includegraphics[width=0.3\linewidth]{capture_1-7}
}
\subfigure[$d=\frac{\pi}{2}$]{
  \includegraphics[width=0.3\linewidth]{capture_1-8}
}
\caption{Plot d}
\end{figure}
\end{task}
\end{document}